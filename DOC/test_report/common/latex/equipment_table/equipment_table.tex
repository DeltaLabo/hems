% equipment_table.tex
%
% Defines a reusable command for creating a table of equipment.
%
% This command, \equipmenttable, takes one argument which is the body of the table.
% The table body should be a series of rows, with each row ending in \\ \hline.
%
% Example usage:
% \equipmenttable{
%   1 & Equipment name & Type or Model & Manufacturer & Calibration Date & Serial No. & Notes \\
%   \hline
%   2 & Oscilloscope & MSO-X 3054A & Keysight & 2024-11-15 & ABC456 & 500 MHz bandwidth \\
%   \hline
% }
\newcommand{\equipmenttable}[1]{
    \begin{tabular}{|p{0.7cm}|p{2.7cm}|p{2.4cm}|p{2.4cm}|p{2.4cm}|p{1.8cm}|p{2cm}|}
    \hline
    \textbf{No.} & \textbf{Equipment Name} & \textbf{Model/Type} & \textbf{Manufacturer} & \textbf{Calibration Date} & \textbf{Serial No.} & \textbf{Notes} \\
    \hline
    #1
    \end{tabular}
}